\documentclass{article}
\newcommand{\version}{\textbf{v0.0 - Template}}
% This document is prepared and maintained by the Roman 
% HLIS Cosmology PIT team.

% Document settings 
\usepackage[english]{babel}
\usepackage[letterpaper,top=2cm,bottom=2cm,left=3cm,right=3cm,marginparwidth=1.75cm]{geometry}
\usepackage{natbib,aas_macros}  % https://ui.adsabs.harvard.edu/help/actions/journal-macros

% Packages
\usepackage{amsmath,amssymb}
\usepackage{graphicx}
\usepackage[colorlinks=true, allcolors=blue]{hyperref}
\usepackage{paralist}
\usepackage{lipsum}

% Custom commands
\newcommand{\pipecal}{{\tt Pipe.M1}}
\newcommand{\pipeid}{{\tt Pipe.M2}}
\newcommand{\pipepz}{{\tt Pipe.M3}}
\newcommand{\pipesample}{{\tt Pipe.M4}}
\newcommand{\pipestats}{{\tt Pipe.M5}}
\newcommand{\pipecluster}{{\tt Pipe.M6}}
\newcommand{\pipecpip}{{\tt Pipe.I1}}
\newcommand{\pipecshear}{{\tt Pipe.I2}}
\newcommand{\pipetxt}{{\tt Pipe.I3}}
\newcommand{\pipeclinf}{{\tt Pipe.I4}}
\newcommand{\pipeadvinf}{{\tt Pipe.I5}}
\newcommand{\Sim}{{\tt Sim}}
\renewcommand{\Roman}{Roman}  

\title{Roman HLIS Cosmology PIT\\
Template Document}
\author{\it Document owner: \it Dida Markovi\v{c}}
\date{\today\\ \version} % See the top of the document for version definition

%%%%%%%%%%%%%%%%%%%%%%%%%%%%%%%%%%%%%%%%%%%%%%%%%%%%%%%%%%%
%%%%%%%%%%%%%%%%%%%%%%%%%%%%%%%%%%%%%%%%%%%%%%%%%%%%%%%%%%%
%%%%%%%%%%%%%%%%%%%%%%%%%%%%%%%%%%%%%%%%%%%%%%%%%%%%%%%%%%%
\begin{document}
\maketitle

\tableofcontents

%%%%%%%%%%%%%%%%%%%%%%%%%%%%%%%%%%%%%%%%%%%%%%%%%%%%%%%%%%%
%%%%%%%%%%%%%%%%%%%%%%%%%%%%%%%%%%%%%%%%%%%%%%%%%%%%%%%%%%%
\newpage
\section{About this document}

\subsection{Purpose}

This document is a template for the \href{https://roman-hlis-cosmology.caltech.edu/}{Roman High-latitude Imaging Survey's Cosmology Project Infrastructure Team} whose work plan is described in our proposal \citep{proposal}. 

\subsection{Scope}

This document can be used as a template for team documents or papers. Feel free to do with it as you please! 

\subsection{Approach aka Instructions}

This is a \TeX\ document owned by the Roman HLIS PIT Pipeline Manager team. It is mirrored as a repository in \href{https://github.com/Roman-HLIS-Cosmology-PIT}{our GitHub Organisation}'s repo in order to facilitate history tracking as well as local editing. It is also a document in Overleaf, which mirrors the GitHub repo.

It is recommended that documents larger than a couple of pages are split into auxiliary files (nicely organised into folders if needed) and that the \texttt{main.tex} file does not contain any or much content text - but that it is reserved mostly for the introductory text (aka this) and \texttt{input} statements. This is especially useful for documents on which many people are collaborating. Read \href{https://www.overleaf.com/learn/latex/Management_in_a_large_project}{more about this on the Overleaf documentation}, if you are interested.

%%%%%%%%%%%%%%%%%%%%%%%%%%%%%%%%%%%%%%%%%%%%%%%%%%%%%%%%%%%
\subsection{Common Acronyms}
%% PLEASE ADD ALPHABETICALLY IF NEEDED

\begin{itemize}
    \item[HLIS] High-latitude Imaging Survey
    \item[PIT] Project Infrastructure Team
    \item[PY] Project Year
    \item[FTE] Full-time equivalent
    \item[WY] Work-year
    \item[PI] Pricipal Investogator
    \item[PM] Pipeline Manager
    \item[SCM] Shear and Clustering Measurement
    \item[CPIP] Cosmological Parameter Inference Pipeline (aka) Cosmological Inference
\end{itemize}

%%%%%%%%%%%%%%%%%%%%%%%%%%%%%%%%%%%%%%%%%%%%%%%%%%%%%%%%%%%
%%%%%%%%%%%%%%%%%%%%%%%%%%%%%%%%%%%%%%%%%%%%%%%%%%%%%%%%%%%
\newpage
\section{Example Section from Input Auxiliary File}
\label{sec:eg}
\lipsum[1]

Also an example citation: \citep[this is inside brackets at the start][and after the citation]{proposal}.

%%%%%%%%%%%%%%%%%%%%%%%%%%%%%%%%%%%%%%%%%%%%%%%%%%%%%%%%%%%
%%%%%%%%%%%%%%%%%%%%%%%%%%%%%%%%%%%%%%%%%%%%%%%%%%%%%%%%%%%
\newpage
\bibliographystyle{unsrtnat}
\bibliography{references}
\end{document}
%%%%%%%%%%%%%%%%%%%%%%%%%%%%%%%%%%%%%%%%%%%%%%%%%%%%%%%%%%%
%%%%%%%%%%%%%%%%%%%%%%%%%%%%%%%%%%%%%%%%%%%%%%%%%%%%%%%%%%%
%%%%%%%%%%%%%%%%%%%%%%%%%%%%%%%%%%%%%%%%%%%%%%%%%%%%%%%%%%%